\begin{itemize}
	\item 市場背景
	\begin{itemize}
		\item 製造業におけるロボティクス需要の高まり (\url{https://prtimes.jp/main/html/rd/p/000002320.000072515.html})
		\begin{itemize}
			\item 2022年 83億米ドル
			\item 2032年 412億米ドル
			\item CAGR 39.5\% → Webサイトは17.4\%の誤植かもしれない($(\frac{412}{83})^{\frac{1}{10}}-1\simeq17.4$)
		\end{itemize}
		\item スマートデバイスの普及 (\url{https://prtimes.jp/main/html/rd/p/000002320.000072515.html})
		\begin{itemize}
			\item ビデオエンターテインメント (\url{https://japan.cnet.com/article/35205930/})
			\begin{itemize}
				\item 2023年 2億7,870万台
				\item 2027年 3億640万台
				\item CAGR 2.4\%
			\end{itemize}
			\item ホームモニタリング・セキュリティ (\url{https://japan.cnet.com/article/35205930/})
			\begin{itemize}
				\item 2023年 1億9,180万台
				\item 2027年 2億5,220万台
				\item CAGR 7.1\%
			\end{itemize}
			\item 照明 (\url{https://japan.cnet.com/article/35205930/})
			\begin{itemize}
				\item 2023年 1億190万台
				\item 2027年 1億6,970万台
				\item CAGR 13.6\%
			\end{itemize}
			\item スマートスピーカ (\url{https://japan.cnet.com/article/35205930/})
			\begin{itemize}
				\item 2023年 1億690万台
				\item 2027年 1億2,250万台
				\item CAGR 3.5\%
			\end{itemize}
			\item 空調 (\url{https://japan.cnet.com/article/35205930/})
			\begin{itemize}
				\item 2023年 2,490万台
				\item 2027年 2,970万台
				\item CAGR 4.5\%
			\end{itemize}
			\item その他 (\url{https://japan.cnet.com/article/35205930/})
			\begin{itemize}
				\item 2023年 1憶5,290万台
				\item 2027年 2憶1,100万台
				\item CAGR 7.7\%
			\end{itemize}
			\item 合計 (\url{https://japan.cnet.com/article/35205930/})
			\begin{itemize}
				\item 2023年 8憶5,710万台
				\item 2027年 10憶9,150万台
				\item CAGR 6.2\%
			\end{itemize}
			\item スマート小売機器市場 (\url{https://newscast.jp/news/0853835})
			\begin{itemize}
				\item 2021年 255憶8,000万米ドル
				\item 2030年 816憶8,000万米ドル
				\item CAGR 15.2\%
			\end{itemize}
			\item スマートカメラ (\url{https://www.gii.co.jp/report/smrc1308705-smart-cameras-market-forecasts-global-analysis-by.html})
			\begin{itemize}
				\item 2023年 35憶4,000万米ドル
				\item 2030年 76憶3,000万米ドル
				\item CAGR 11.57\%
			\end{itemize}
		\end{itemize}
		\item 自動運転市場の拡大 (\url{https://prtimes.jp/main/html/rd/p/000002320.000072515.html})
		\begin{itemize}
			\item 自動運転(レベル別台数) (\url{https://monoist.itmedia.co.jp/mn/articles/2208/04/news070.html})
			\begin{itemize}
				\item レベル2 2022年 3,608万台
				\item レベル2 2030年 6,176万台
				\item レベル2 2045年 6,166万台
				\item レベル3 2022年 3万台
				\item レベル3 2030年 580万台
				\item レベル3 2045年 2,847万台
				\item レベル4/5 2022年 9万台
				\item レベル4/5 2030年 433万台
				\item レベル4/5 2045年 2,051万台
			\end{itemize}
			\item 自動運転レベル(参考) (\url{https://jidounten-lab.com/autonomous-level})
			\begin{itemize}
				\item 自動運転レベル1:運転支援
				\item 自動運転レベル2:部分運転自動化
				\item 自動運転レベル3:条件付き運転自動化
				\item 自動運転レベル4:高度運転自動化
				\item 自動運転レベル5:完全運転自動化
			\end{itemize}
			\item 自動運転(地域別台数) (\url{https://monoist.itmedia.co.jp/mn/articles/2208/04/news070.html})
			\begin{itemize}
				\item 日本 2022年 1万台
				\item 日本 2030年 97万台
				\item 日本 2045年 377万台
				\item 欧州 2022年 2万台
				\item 欧州 2030年 306万台
				\item 欧州 2045年 1,294万台
				\item 北米 2022年 5万台
				\item 北米 2030年 278万台
				\item 北米 2045年 1,042万台
				\item 中国 2022年 4万台
				\item 中国 2030年 237万台
				\item 中国 2045年 1,411万台
			\end{itemize}
		\end{itemize}
		\item 低遅延 (\url{https://www.mordorintelligence.com/ja/industry-reports/edge-ai-hardware-market})
		\item 運用コスト削減 (\url{https://www.mordorintelligence.com/ja/industry-reports/edge-ai-hardware-market})
	\end{itemize}
	\item エッジAIの特徴と活用シーン (\url{https://www.toshiba-dme.co.jp/dme/dig/edge_solution/trend.htm})
	\begin{itemize}
		\item リアルタイム性
		\begin{itemize}
			\item ネットワークを介さないデータ処理で低遅延に実行可能
			\item 製造現場の不良品検出や自動運転などでの活用が期待される
		\end{itemize}
		\item 情報漏洩リスクの低減
		\begin{itemize}
			\item クラウド利用がない為,機密情報の漏洩リスクが低減できる
			\item 防犯カメラ・金融サービス・入退室管理などでの活用が期待される
		\end{itemize}
		\item 通信負荷とコスト削減
		\begin{itemize}
			\item データ処理がエッジで完結する,またはクラウドへアップロードするデータ量削減による通信量削減で,コスト低減
			\item 農業や水産業などの通信環境の良くないシーンでの活用が期待される
		\end{itemize}
		\item 低消費電力
		\begin{itemize}
			\item 用途・要求に最適化したソリューションで消費電力を削減(対パソコンなど)
		\end{itemize}
		\item デバイスの軽薄短小化
		\begin{itemize}
			\item 用途・要求に最適化したソリューションで設置面積を削減(対パソコンなど)
		\end{itemize}
		\item 低遅延 (\url{https://aismiley.co.jp/ai_news/edge-ai-software-market-expects-20-3-growth-by-2027/})
	\end{itemize}
	\item 市場規模・予測
	\begin{itemize}
		\item エッジAI市場
			\item 日本 (\url{https://prtimes.jp/main/html/rd/p/000002320.000072515.html})
			\begin{itemize}
				\item 2022年 81.2憶米ドル
				\item 2032年 421億米ドル
			\end{itemize}
			\item AIビジネス国内市場 (\url{https://www.fuji-keizai.co.jp/press/detail.html?cid=19039})
			\begin{itemize}
				\item 2018年度 5,301億円
				\item 2022年度 1兆2,000億円
				\item 2030年度 2兆1,286億円
			\end{itemize}
			\item エッジAIコンピューティング市場 (\url{https://www.fuji-keizai.co.jp/press/detail.html?cid=19039})
			\begin{itemize}
				\item 2018年 110億円
				\item 2030年 664億円
			\end{itemize}
			\item エッジAI市場 (\url{https://it.impress.co.jp/articles/-/24005})
			\begin{itemize}
				\item 2021年度 76.6億円
				\item 2022年度 117億円
				\item 2026年度 431億円
			\end{itemize}
			\item 国内AIシステム市場 (\url{https://aismiley.co.jp/ai_news/idc-japan-2023/})
			\begin{itemize}
				\item 2023年 4,930億7,100万円
				\item 2027年 1兆1,034憶7,700万円
				\item CAGR 23.2\%
			\end{itemize}
		\item エッジコンピューティング市場 (\url{http://www.ex-press.jp/lfwj/lfwj-news/lfwj-biz-market/52519/})
		\begin{itemize}
			\item 2030年 1559憶米ドル
			\item 2023-2030年 CAGR 37.9\%成長
		\end{itemize}
		\item エッジAIコンピューティング市場 (\url{http://ex-press.jp/lfwj/lfwj-news/lfwj-biz-market/45075/})
		\begin{itemize}
			\item 2020年 90億9600万米ドル
			\item 2030年 596億3300万米ドル
			\item CAGR 21.2\%成長
		\end{itemize}
		\item エッジAIチップ (\url{https://newscast.jp/news/4431496})
		\begin{itemize}
			\item 2030年 132憶米ドル
		\end{itemize}
		\item エッジAIプロセッサ (\url{https://prtimes.jp/main/html/rd/p/000002320.000072515.html})
		\begin{itemize}
			\item 2022年 27.5憶米ドル
			\item 2032年 110憶米ドル
			\item CAGR 15.56\%
		\end{itemize}
		\item エッジAIプロセッサ タイプ別内訳 (\url{https://prtimes.jp/main/html/rd/p/000002320.000072515.html})
		\begin{itemize}
			\item 中央処理装置:CPU
			\item グラフィック処理装置:GPU
			\item 特定用途向け集積回路:ASIC
			\begin{itemize}
				\item 2022年 4億米ドル
				\item 2032年 16億米ドル
			\end{itemize}
		\end{itemize}
		\item エッジAIソフトウェア市場 (\url{https://www.value-press.com/pressrelease/267210})
		\begin{itemize}
			\item 2020年 5億9000万米ドル
			\item 2026年 18億3500万米ドル
			\item CAGR 20.8\%
		\end{itemize}
	\end{itemize}
	\item エッジAIプロセッサ市場の成長を制限する課題 (\url{https://prtimes.jp/main/html/rd/p/000002320.000072515.html})
	\begin{itemize}
		\item 半導体不足
		\item AIテクノロジーの成長に伴う実装コストの増加
	\end{itemize}
	\item エッジAIソリューションの成長を制限する課題 (\url{https://www.value-press.com/pressrelease/267210})
	\begin{itemize}
		\item プライバシー
		\item セキュリティ
		\begin{itemize}
			\item セキュリティ侵害・攻撃・DDoSの増加により,当該分野ではエッジAIの採用を制限
		\end{itemize}
	\end{itemize}
	\item エッジAIの注目トレンド
	\begin{itemize}
		\item \url{https://blogs.nvidia.co.jp/2023/01/17/edge-ai-trends-2023/}
	\end{itemize}
	\item ベンチマーク
	\begin{itemize}
		\item エッジデバイス向けAIアクセラレータの比較検証 (\url{https://lab.mo-t.com/blog/benchmark-edge-ai-accelarator})
	\end{itemize}
	\item 関連団体
	\begin{itemize}
		\item tinyML (\url{https://www.tinyml.org/})
		\item Edge AI and Vision Alliance (\url{https://www.edge-ai-vision.com/})
	\end{itemize}
	\item 関連アプリケーション (\url{https://www.value-press.com/pressrelease/267210,https://www.mapion.co.jp/news/release/dn0000270836-all/},\url{https://news.mynavi.jp/techplus/article/20211214-2227346/})
	\begin{itemize}
		\item リモートワーク
		\item リモート監視
		\item プラントオートメーション
		\item 遠隔医療
		\item スマートホーム
		\item 監視カメラ
		\item 予知保全
		\item 遠隔計測
		\item 混雑状況監視(人,車)
	\end{itemize}
	\item エッジAI活用事例:小売店 (\url{https://standard-dx.com/post_blog/edge-ai})
	\begin{itemize}
		\item 課題
		\begin{itemize}
			\item 購買時のデータ不足,欠品などによる販売機会の損失
			\item ニーズの多様化
			\item レジ打ち,売り場確認などによる人件費増加
		\end{itemize}
		\item ソリューション
		\begin{itemize}
			\item リテールAIカメラ
			\begin{itemize}
				\item 購入順序の把握
				\item 棚前行動の分析
				\item 滞在時間の把握
				\item 欠品状況監視
				\item 補充タイミングの通知
				\item 棚割最適化
			\end{itemize}
			\item スマートショッピングカート
			\begin{itemize}
				\item レジレス・キャッシュレス
				\item レコメンド・レシピ提案
			\end{itemize}
		\end{itemize}
		\item 導入効果
		\begin{itemize}
			\item スマートショッピングカート
			\begin{itemize}
				\item 有人レジ(125秒),無人レジ(75秒)に対して,スマートショッピングカートで32秒までレジの所要時間を短縮
				\item 来店頻度が13.8\%増加
			\end{itemize}
			\item リテールAIカメラ
			\begin{itemize}
				\item 棚回転効率が28.7\%向上
				\item 欠品率が12.2\%改善
			\end{itemize}
		\end{itemize}
		\item 参考
		\begin{itemize}
			\item \url{http://www.trial-net.co.jp/cp/mediakit_ssc_aicamera/}
		\end{itemize}
	\end{itemize}
	\item エッジAI活用事例:美容室・アパレル店舗 (\url{https://standard-dx.com/post_blog/edge-ai})
	\begin{itemize}
		\item 課題
		\begin{itemize}
			\item 化粧品の色味を探す工程にかかる時間と手間
			\item 肌へのダメージ
			\item マーケティング
		\end{itemize}
		\item 解決策
		\begin{itemize}
			\item 明暗やアンダートーンを含む約90,000種類の肌色データを学習したDeepLearningモデルで利用者の肌に最適な色味の検出とレコメンドやバーチャルメイクを行う
			\item オフラインはスマートミラーで,オンラインはスマートフォンやパソコンで利用 (スマートシェードファインダー)
		\end{itemize}
		\item 導入効果
		\begin{itemize}
			\item トライされた商品の売上が2.5倍上昇
			\item スマートシェードファインダー利用による顧客ロイヤリティアップ
		\end{itemize}
		\item 参考
		\begin{itemize}
			\item \url{https://www.perfectcorp.com/business/successstory/detail?id=6}
		\end{itemize}
	\end{itemize}
	\item エッジAI活用事例:生産ライン (\url{https://standard-dx.com/post_blog/edge-ai})
	\begin{itemize}
		\item 課題
		\begin{itemize}
			\item 巻き線機でフィルムの特性変化や付け替えによる固有振動数の変化に伴う振動抑制制御補正にかかる時間が長い(PID制御では振動抑制するために約10秒)
		\end{itemize}
		\item 解決策
		\begin{itemize}
			\item 固有振動の予測と自動補正(DeepBinaryTreeアルゴリズムを採用)
		\end{itemize}
		\item 効果
		\begin{itemize}
			\item 機器特性変化に対する補正を3秒まで短縮(10秒→3秒への7秒の短縮)
			\item 約20m発生していた不良品が1/3以下に現象
		\end{itemize}
		\item 参考
		\begin{itemize}
			\item \url{https://aising.jp/algorithms/}
		\end{itemize}
	\end{itemize}
	\item エッジAI活用事例:生産ライン (\url{https://www.toshiba-dme.co.jp/dme/dig/edge_solution/trend.htm})
	\begin{itemize}
		\item 自動化,省人化
		\begin{itemize}
			\item 課題
			\begin{itemize}
				\item 検査精度がバラつく
				\item 不良品が多い
				\item 危険
			\end{itemize}
			\item 効果
			\begin{itemize}
				\item 検査精度の均一化
				\item 歩留まり向上
				\item 安全性向上
				\item 人件費削減
			\end{itemize}
		\end{itemize}
		\item 熟練スタッフの技術継承
		\begin{itemize}
			\item 課題
			\begin{itemize}
				\item 属人化している工程がある
				\item 熟練スタッフの退職によりノウハウが損失する
			\end{itemize}
			\item 効果
			\begin{itemize}
				\item 判断が自動化
				\item 省人化
				\item 品質向上
			\end{itemize}
		\end{itemize}
		\item 立ち入り禁止エリアの人検知
		\begin{itemize}
			\item 課題
			\begin{itemize}
				\item 危険を即時に検知できない
				\item 安全対策コストが高い
				\item 稼働率が低下する
			\end{itemize}
			\item 効果
			\begin{itemize}
				\item 緊急時の即応
				\item 稼働率と安全性の両立
				\item エリア内の異常検知を自動化
			\end{itemize}
		\end{itemize}
		\item 設備の予知保全
		\begin{itemize}
			\item 課題
			\begin{itemize}
				\item 設備が突然故障する
				\item 定期保全作業費用が高い
			\end{itemize}
			\item 対策
			\begin{itemize}
				\item 稼働率を安定化できる
				\item 設備保全人件費の低減
			\end{itemize}
		\end{itemize}
	\end{itemize}
	\item 関連企業 (\url{https://newscast.jp/news/4431496},\url{https://www.mordorintelligence.com/ja/industry-reports/edge-ai-hardware-market})
	\begin{itemize}
		\item Advanced Micro Devices, Inc., Xilinx, Inc.
		\begin{itemize}
			\item Kria KV260 ビジョン AI スターター キット (\url{https://japan.xilinx.com/products/som/kria/kv260-vision-starter-kit.html})
			\item Vitis AI (\url{https://japan.xilinx.com/products/design-tools/vitis/vitis-ai.html})
			\begin{itemize}
				\item AI推論開発プラットフォーム
				\item PyTorch,TensorFlow,ONNXフォーマットの学習済みモデルをAI Model Zooで提供
				\item AIモデル構造を圧縮するVitis AI Optimizer
				\item AIモデルの重みを量子化するVitis AI Quantizer
				\item AIモデルを高効率な命令セットとデータフローへのマッピングやレイヤー融合・スケジューリングなどの最適化を行うVitis AI Compiler
				\item AI推論実装における効率性と使用率を分析可能なVitis AI Profiler
				\item 高効率な推論処理を行うためのVitis AI Library
			\end{itemize}
			\item Vivado ML
			\begin{itemize}
				\item 機械学習向けFPGA開発環境
			\end{itemize}
			\item 
		\end{itemize}
		\item Alphabet Inc. (Google LLC)
		\begin{itemize}
			\item Edge TPU (\url{https://cloud.google.com/edge-tpu?hl=ja})
		\end{itemize}
		\item Intel Corporation
		\begin{itemize}
			\item FPGA (\url{https://www.intel.co.jp/content/www/jp/ja/products/docs/programmable/ai-fpga-whitepaper.html})
			\item Movidius VPU (\url{https://www.intel.co.jp/content/www/jp/ja/products/details/processors/movidius-vpu.html})
			\item OpenVINO (\url{https://www.intel.co.jp/content/www/jp/ja/internet-of-things/openvino-toolkit.html})
		\end{itemize}
		\item Qualcomm Technologies, Inc.
		\begin{itemize}
			\item Edge AI Box (\url{https://www.qualcomm.com/products/technology/artificial-intelligence/edge-ai-box})
		\end{itemize}
		\item Apple Inc.
		\begin{itemize}
			\item Core ML (\url{https://developer.apple.com/jp/machine-learning/core-ml/})
		\end{itemize}
		\item Mythic
		\begin{itemize}
			\item Mythic Analog Matrix Processors (\url{https://mythic.ai/})
		\end{itemize}
		\item ARM Ltd.
		\begin{itemize}
			\item Cortex CPU (\url{https://www.arm.com/ja/product-filter?families=cortex-a&families=cortex-m&use-cases=machine%20learning})
			\item Ethos NPU (\url{https://www.arm.com/ja/product-filter?families=ethos%20npus})
			\item Mali GPU (\url{https://www.arm.com/ja/product-filter?families=mali%20image%20signal%20processors,mali%20graphics%20processors&use-cases=machine%20learning&showall=true})
		\end{itemize}
		\item SAMSUNG ELECTRONICS CO., LTD.
		\begin{itemize}
			\item あまり注力してなさそう (\url{https://semiconductor.samsung.com/jp/solutions/applications/ai/})
		\end{itemize}
		\item NVIDIA Corporation
		\begin{itemize}
			\item エッジでのAI向けソリューション (\url{https://www.nvidia.com/ja-jp/deep-learning-ai/solutions/ai-at-the-edge/})
			\item Jetson Nano (\url{https://www.nvidia.com/ja-jp/autonomous-machines/embedded-systems/jetson-nano/product-development/})
			\item Jetson AGX Xavier (\url{https://www.nvidia.com/ja-jp/autonomous-machines/embedded-systems/jetson-agx-xavier/})
			\item Jetson AGX Orin Industrial (\url{https://developer.nvidia.com/ja-jp/blog/step-into-the-future-of-industrial-grade-edge-ai-with-nvidia-jetson-agx-orin-industrial/})
		\end{itemize}
		\item HiSilicon Technology Co., Ltd
		\begin{itemize}
			\item HUAWEI Ascend 310 (\url{https://www.hisilicon.com/en/products/Ascend/Ascend-310})
			\item HUAWEI Ascend 910 (\url{https://www.hisilicon.com/en/products/Ascend/Ascend-910})
		\end{itemize}
		\item Huawei Technologies Co.、Ltd.
		\begin{itemize}
			\item Atlas 500 AIエッジステーション (\url{https://e.huawei.com/jp/products/cloud-computing-dc/atlas/atlas-500})
		\end{itemize}
		\item MediaTek Inc.
		\begin{itemize}
			\item MediaTek AI プロセッシングユニット(APU) (\url{https://www.mediatek.jp/innovations/artificial-intelligence})
			\item 6th Generation MediaTek APU (AI for Smartphones) (\url{https://www.mediatek.com/technology/ai-for-smartphones-6th-gen})
			\item 5th Generation MediaTek APU (AI for Smartphones) (\url{https://www.mediatek.com/technology/ai-for-smartphones-5th-gen})
		\end{itemize}
		\item Imagination Technologies Limited
		\begin{itemize}
			\item PowerVR, Neural Network Accelerators (\url{https://www.imaginationtech.com/products/ai/})
		\end{itemize}
		\item IBM Corporation
		\begin{itemize}
			\item Edge AI SDK (\url{https://researcher.watson.ibm.com/researcher/view_group.php?id=10944})
		\end{itemize}
		\item Amazon Web Services, Inc.
		\begin{itemize}
			\item Amazon SageMaker Edge (\url{https://aws.amazon.com/jp/sagemaker/edge/?sagemaker-data-wrangler-whats-new.sort-by=item.additionalFields.postDateTime&sagemaker-data-wrangler-whats-new.sort-order=desc})
			\item AI /機械学習に活用できるAWSのエッジソリューションのご紹介 (\url{https://pages.awscloud.com/rs/112-TZM-766/images/EV_awsloft-tko-iotloft14-lt4_Sep-2020.pdf})
		\end{itemize}
		\item Synaptics Incorporated
		\begin{itemize}
			\item Edge Computing SoCs with AI (\url{https://www.synaptics.com/technology/edge-computing})
			\item Synaptics Edge AI (\url{https://investor.synaptics.com/static-files/46bac204-81d8-45e1-a6b1-0e437c795bf9})
		\end{itemize}
		\item imagimob
		\begin{itemize}
			\item Development Platform for Machine Learning on Edge devices (\url{https://www.imagimob.com/})
		\end{itemize}
		\item TIBCO Software Inc.
		\begin{itemize}
			\item TIBCO Spotfire (\url{https://www.tibco.com/products/tibco-spotfire})
			\item Make AI Real: Operationalize Data Science (\url{https://www.tibco.com/products/data-science})
			\item Bringing Intelligence to Edge Computing Through Machine Learning (\url{https://www.tibco.com/resources/whitepaper/bringing-intelligence-edge-computing-through-machine-learning})
		\end{itemize}
		\item Octonion
		\begin{itemize}
			\item つぶれた?Webサイトにアクセスできない (\url{https://prtimes.jp/main/html/rd/p/000001113.000001337.html})
		\end{itemize}
		\item FogHorn System
		\begin{itemize}
			\item Johnson Controlsに買収されたらしいが,その後どうなったかが分からない (\url{https://prtimes.jp/main/html/rd/p/000000127.000007712.html})
		\end{itemize}
		\item Gorilla Technology Group
		\begin{itemize}
			\item エッジAIテクノロジー (\url{https://jp.gorilla-technology.com/Products/Technology/Edge-AI})
		\end{itemize}
		\item STMicroelectronics
		\begin{itemize}
			\item Edge AI - Artificial Intelligence at the Edge (\url{https://www.st.com/content/st_com/ja/campaigns/artificial-intelligence-at-the-edge.html})
			\item X-CUBE-AI (\url{https://www.st.com/ja/embedded-software/x-cube-ai.html})
			\item X-LINUX-AI (\url{https://www.st.com/ja/embedded-software/x-linux-ai.html})
			\item FP-AI-SENSING1 (\url{https://www.st.com/ja/embedded-software/fp-ai-sensing1.html})
			\item FP-AI-VISION1 (\url{https://www.st.com/ja/embedded-software/fp-ai-vision1.html})
			\item FP-AI-NANOEDG1 (\url{https://www.st.com/ja/embedded-software/fp-ai-nanoedg1.html})
		\end{itemize}
		\item Renesas Electronics Corporation.
		\begin{itemize}
			\item 人工知能(AI) (\url{https://www.renesas.com/jp/ja/application/key-technology/artificial-intelligence})
		\end{itemize}
		\item Texas Instruments Incorporated.
		\begin{itemize}
			\item エッジインテリジェンスの推進 (\url{https://www.ti.com/ja-jp/technologies/edge-ai.html})
			\item AM68A (\url{https://www.ti.com/product/ja-jp/AM68A})
			\item AM69A (\url{https://www.ti.com/product/ja-jp/AM69A})
			\item AM62A3 (\url{https://www.ti.com/product/ja-jp/AM62A3})
			\item AM62A7 (\url{https://www.ti.com/product/ja-jp/AM62A7})
			\item EDGE-AI-STUDIO (/url{https://www.ti.com/tool/ja-jp/EDGE-AI-STUDIO})
		\end{itemize}
		\item Idein Inc.
		\begin{itemize}
			\item 国内シェアNo.1 エッジAIプラットフォーム Actcast (\url{https://www.idein.jp/ja})
		\end{itemize}
	\end{itemize}
\end{itemize}

